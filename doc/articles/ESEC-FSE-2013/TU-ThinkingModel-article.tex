\documentclass{acm_proc_article-sp}

\begin{document}

\title{Thinking model and machine understanding of English primitive texts and it's application in Infrastructure as Service domain.}

\numberofauthors{3}
\author{
% 1st. author
\alignauthor Alexander Toschev\\
       \affaddr{Kazan State University}\\
       \affaddr{Universitetskaya 17,}\\
       \affaddr{420008 Kazan, Russia}
       \email{alexander.toschev@gmail.com}
% 2nd. author
\alignauthor Max Talanov\\
       \affaddr{Fujitsu GDC Russia, Kazan, Russia}\\
       \affaddr{Sibirskii trakt 34,}\\
       \affaddr{420029 Kazan, Russia}
       \email{max.talanov@gmail.com}
% 3rd. author
\alignauthor Andrey Krekhov\\
       \affaddr{Fujitsu GDC Russia, Kazan, Russia}\\
       \affaddr{Sibirskii trakt 34,}\\
       \affaddr{420029 Kazan, Russia}
       \email{andrey.krekhov@ts.fujitsu.com}
}

\date{22 February 2013}

\maketitle

\begin{abstract}

Construction of machine understanding is definitely the challenge. There are several technologies used widely.
Currently mainstream applications uses machine operatable knowledge bases, for example Wolfram Alpha to support simple dialog and operate devices.
Newer the less those approaches do not answer the question how do machine could be capable to understand human created text.
We tried new approach based on assumption that human understanding is tightly coupled with human thinking itself.
We used thinking model described in Marvin Minsky book "The emotion machine".

\end{abstract}

\section{The emotion machine thinking model}
\subsection{6 thinking levels}
\subsection{Selector, Critic, Way to think triple}

\section{Implementation of thinking model}
\subsection{Thinking levels control}
\subsection{Memory: short term, long term}


\section{IS domain application of the thinking model}
\subsection{Critics and ways to think for incident processing}

\section{Practical results}
\subsection{Direct instruction processing}
\subsection{Problem description processing}


\subsection{References}

\end{document}